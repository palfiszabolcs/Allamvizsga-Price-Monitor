Röviden összefoglalva a fentebb tárgyaltakat, el lehet mondani, hogy sok esetben megalapozottan ítéljük meg az egyes webáruházakat az árak tudatos manipulálása miatt, viszont mint mindig, ebben az esetben sem lehet általánosítani, hiszen az adatok elemzése során találkoztunk valós kedvezményekkel is, nem csak gyanúsan változó árakkal, melyek félrevezetik vagy egyenesen becsapják a vásárlót. Ezeket figyelembevéve, érdemes mindig több ideig figyelni valamit, ami iránt esetleg vásárlási szándékunk van és nem szabad pusztán az eladóra hagyatkozni, amikor kiírja a termék régi árát, egy nagyobb árleszállítás keretein belül. Erre a célra lett megtervezve és megvalósítva a dolgozatban bemutatott szoftver mely lehetőséget nyújt a felhasználóknak vagy potenciális vásárlóknak automatizálni az esetleges kívánt termékek napi ellenőrzését és az árak összehasonlítását. Mindezt megteheti bárhol is tartózkodik, a megvalósított telefonos alkalmazás segítségével, de egy egyszerűen használható böngészős kiegészítőből is, ha netán inkább az áll közelebb hozzá.

Természetesen egy ilyen típusú alkalmazás fejlesztése folyamatos munkát igényel. Van helye további fejlődésnek, változásnak, optimalizálásnak a jövőre nézve. Mivel a webáruházak struktúrája folyamatosan változik, ezért a háttérben működő szoftvert folyamatosan igazítani kell. Továbbá, egy lehetséges fejlesztés az új webshopok támogatása a jövőben, úgy hazai mint külföldi oldalakat figyelembe véve. Felhasználói felület szintjén is van hely fejlődésnek vagy új akár funkciók hozzáadásának, melyet majd idővel meg is lehet tenni.

\section{Megvalósítások}

A dolgozatban leírtak alapján a következők lettek megvalósítva:
\begin{itemize}
    \item Web Scraping tanulmányozása szakirodalom által
    \item Web Scraping jogi és etikai háttereinek tanulmányozása
    \item Web áruházak struktúrájának tanulmányozása
    \item Az elkészítendő rendszer specifikálása
    \item Firebase adatbázis be üzemelése
    \item Web Scraper implementálása Python script által
    \item Böngésző kiegészítő megvalósítása
    \item Telefonos alkalmazás megvalósítása Flutter segítségével
    \item Admin felület megvalósítása PyQt5 segítségével
    \item Esettanulmány az árak változásáról, külön kitérve a 2020-as Black Friday periódusra
\end{itemize}