A mai rohanó világban a bevásárlások egyre növekvő százaléka történik az interneten, mindez lehetőséget nyújtva a vásárlóknak, hogy egy bizonyos terméket több, akár hazai akár külföldi, oldalról is megvásárolhasson. Az e-commerce-el foglalkozó cégek rohamos fejlődésnek indultak az utóbbi évtizedben mely maga után vonja az érdekesebbnél érdekesebb marketing fogásokat, melyekkel a célközönséget próbálják vásárlásra bírni.

Valószínűleg mindenki hallott már a “Black Friday” az-az „Fekete Péntek” -nek nevezett jelenségről amely inspirációként szolgált az alkalmazás megvalósításához. Ez a kifejezés legelőszőr az 1800-as években fogalmazódott meg, amikor is Jay Gould és James Fisk az amerikai arany árak manipulálása által 20\%-os esést okoztak a részvénypiacon melynek következtében az árucikkek értéke felére csökkent. A 20. század közepe fele ez már egészen más jelentéssel bírt, ugyanis a Hálaadás ünnepét követő napon, az-az pénteken vette kezdetét a karácsonyi árleszállítás, mely sok cég esetében életmentő volt, hiszen ekkor kerültek át a veszteséges állapotból melyet pirossal jelöltek, a jövedelmezőbe, amit már fekete írószerrel jegyeztek fel. Ebben az időszakban a megszokottnál jóval nagyobb és több árleszállítással vonzották az embereket.

Mint azt sokan tudjuk, országunkban is nagy népszerűségnek örvend ez a jelenség, habár eléggé távol áll az eredeti koncepciótól. Nagyon sok mesterséges árleszállítással próbálják becsapni az embert, melyet legtöbb esetben jól kitervelt ár ingadozással oldanak meg. Ugyanakkor, nem kizárólag ebben a periódusban lehet észrevenni az úgymond „hamis” kedvezményeket ezért szükségét láttuk egy olyan alkalmazás kifejlesztésének, amely nyomon tudja követni egy megadott termék árat, illetve annak ingadozását.

Mivel az interneten publikus adatok találhatók, ezek felhasználásával semmiéle kár nem keletkezik az adott weboldalak számára. Azt a HTML kódot, amelyből az információkat kinyerni szeretnénk, bármelyik felhasználó megtekintheti az F12 billentyű megnyomásával vagy az adott oldalon jobb click-et követően az “Inspect” opcióra kattintva. Az alkalmazás ezen nyilvános adatok feldolgozásával és elemzésével kivan foglalkozni.
