\section{Felhasználói Követelmények}

A felhasználónak mindenekelőtt be kell tudnia jelentkezni, ahhoz, hogy elérje a terméklistáját, ez azért fontos, mivel eszköz váltás esetén nem szeretnénk elveszíteni az addig követett termékeket. A bejelentkezéshez szükséges adatok a regisztrációkor megadott email cím, illetve jelszó. Mindezt az \ref{fig:login_activity_diag} ábra illusztrálja. A felhasználó jelszava base64 hash általi titkosítással van eltárolva, ezáltal az nem elérhető eredeti formájában, ezt a funkciót a bejelentkezést megvalósító Firebase Authentication valósítja meg, amely jelszó módosítási lehetőséget is lehetővé tesz.

\begin{figure}[h]
    \centering
    \includegraphics[scale=0.5, width=\textwidth]{figures/images/login_activity.png}
    \caption{Login Activity}
    \label{fig:login_activity_diag}
\end{figure}

Abban az esetben, ha a felhasználó nincs regisztrálva, megteheti ezt a „ Not registered? Click here! ” szövegre kattintva. A regisztrációhoz szükséges egy érvényes e-mail cím, valamint jelszó megadása. Sikeres regisztrálás esetén a felhasználó vissza kerül a bejelentkező ablakra, ahol be tud lépni, azzal a feltétellel, hogy az automatikusan küldött levél által visszaigazolta email címét.

A bejelentkezést követően a felhasználó a főoldalra kerül, ahol a követett termékek listáját tekintheti meg. A listában minden egyes elemnek látható a megnevezése, egy a terméket ábrázoló kép, illetve az adott termék aktuális ára. Továbbá ezen az oldalon lehetősége van a felhasználónak új termékeket hozzá adni a listához, lásd \ref{fig:dashboard_activity_diag} ábra.

\begin{figure}[H]
    \centering
    \includegraphics[scale=0.3]{figures/images/dashboard_activity.png}
    \caption{Dashboard Activity}
    \label{fig:dashboard_activity_diag}
\end{figure}

Egy a listában lévő elemre kattintva, az alkalmazás átvisz egy másik oldalra, ahol több információt kapunk a követett termékről. A termék árát tartalmazó gombra kattintva, egy grafikonon tekinthetjük meg a termék árának változását a hozzáadás napjától az aktuális dátumig, lásd \ref{fig:chart_example} ábra. Az „ See product page ” gombra kattintva az alkalmazás megnyitja a terméket tartalmazó weboldalt. Ugyanitt található a törlés gomb, melyre kattintva a termek törlésre kerül a listából és nem fogjuk tovább követni, lásd \ref{fig:detailed_view_diag} ábra.

\begin{figure}[H]
    \centering
    \includegraphics[scale=0.3]{figures/images/chart_view_activity.png}
    \caption{Detailed View}
    \label{fig:detailed_view_diag}
\end{figure}

\begin{figure}[H]
    \centering
    \includegraphics[scale=1]{figures/images/chart_example.png}
    \caption{Chart of Price Change}
    \label{fig:chart_example}
\end{figure}

A felhasználó rendelkezésére áll továbbá két menü. Egy információs, amely röviden leírja az alkalmazás használatát, illetve tartalmazza az általa támogatott weboldalak listáját. A másik a felhasználó fiókjával kapcsolatos információkat és funkciókat tartalmaz. Itt tekintheti meg a felhasználó, hogy milyen email címmel jelentkezett be, megváltoztathatja az aktuális jelszavát, törölheti a fiókját, illetve kijelentkezhet az alkalmazásból, lásd \ref{fig:additional_info} ábra.

\begin{figure}[H]
    \centering
    \includegraphics[scale=0.3]{figures/images/additional_features.png}
    \caption{Additional information}
    \label{fig:additional_info}
\end{figure}

\section{Rendszer Követelmények}

\subsection{Funkcionális követelmények}

A rendszernek mindenek elött, egy bejelentkezési, illetve, regisztrálási felülettel kell rendelkeznie. Regisztrálás után, a felhasználónak egy ellenőrző email-t kell kapnia, amivel igazolja, hogy ő a cím tulajdonosa. A bejelentkezés nem lehetséges, abban az esetben, ha a felhasználó nem igazolta vissza az előbb említett email-ben a címét. A cím igazolása egy linkre való kattintással történik.

A felhasználónak lehetősége van a jelszavának módosítására melyet a bejelentkezési felületről ér el. Miután a felhasználó beírta az email címét, egy levelet fog kapni az adott címre, amelyen keresztül új jelszót tud beállítani.

Bejelentkezést követően, a felhasználó egy felületet lát, melyen bizonyos műveleteket végezhet. Megtekintheti a profiljához tartozó email címét, valamint törölheti a felhasználóját. Ugyanakkor, lehetősége van kijelentkezni az alkalmazásából melynek hatására újra a bejelentkezési oldalra kerül.

Ugyancsak a főoldalról a felhasználónak lehetősége van az alkalmazás használatával kapcsolatos információk megtekintésére mely tartalmaz egy listát is. A lista bizonyos weboldalkát tartalmaz, melyeket kiválasztva, az alkalmazás átirányít az adott elem oldalára.

A felhasználónak lehetősége van termékeket hozzáadni és kitörölni a listájából, valamint görgetni a lista tartalmában. A terméklistában egy elemet kiválasztva, részletes reprezentációt kap az adott elem tárolt adatairól.

\subsection{Nem-Funkcionális követelmények}

% TODO
TODO...
