These days the majority of shopping is done online. There are tons of online shopping websites, with some kind of big sale going on almost all the time, however these sales are most of the time made up. These false sales can be spotted only if you follow a products price daily, for a longer period of time which is a repetitive and time-consuming process. The same can be said when you’re interested in purchasing some products but you’re not quite decided yet, so you want to wait for the price to drop. These gave the inspiration for developing our system.

In this paper we’ll present a price monitoring system which automates the process described earlier. At predefined moments the system wakes up and scrapes publicly available data from the Internet, which are than saved. The user has the opportunity to add products to his list, for them to be followed, and can view the price change on a chart. All this can be done from a Chrome browser extension or a mobile application built with Flutter. The so-called heart of the system is a python script which is responsible for the mining of the data, periodically updating the prices and pushing new products into the database. All this is done while respecting all moral and legal boundaries. All of our data is stored in a Realtime Database provided by Firebase, which turned out to be perfect for our application.

After months of gathering data, while using the system, we analyzed it and made a few conclusions. We saw multiple patterns emerging from the price changes but there is not one that matches every product. There were many scams regarding prices, especially during Black Friday but there were also real and significant price drops as well. After using the system for a longer period of time we can say that a software like this helps the buyers in making a purchase decision, by providing useful data in a easy to read form.

\textbf{Keywords}: webscraping, browser extension, Flutter app