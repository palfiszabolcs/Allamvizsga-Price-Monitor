În zilele noastre, majoritatea cumpărăturilor se fac online. Sunt foarte multe magazine online, care iți promit diferite promoții, indiferent de perioada anului. Aceste promoții de multe ori sunt doar de fațadă, în spatele lor de fapt nu este nici o scădere reală de preț. Aceste aspecte se pot observa doar în cazul în care urmărim prețul unui produs zilnic, ceea ce este foarte repetitiv și solicitant în raport cu timpul nostru. Toate aceste aspecte sunt valabile și în cazul în care dorim să cumpăram produsele la prețuri corecte și cu adevărat aflate la promoție. 

In această lucrare este prezentat un sistem pentru monitorizarea prețurilor care automatizează procesul descris mai devreme. Sistemul se trezește in momente predefinite pentru a mina date publice de pe Internet, care apoi vor fi salvate. Utilizatorul are la dispoziție o extensie de browser și o aplicație pentru telefon realizată în Flutter, prin care poate adăuga produse noi în listă și poate urmări pe un grafic, evoluția preturilor. Așa zisa inimă a sistemului o reprezintă un script realizat în Python, care este responsabil pentru adunarea datelor, actualizarea periodică a prețurilor și a adăugării unor noi produse în lista utilizatorului. Datele adunate sunt stocate într-o bază de date furnizată de Firebase, care s-a dovedit a fi ideală pentru utilizarea noastră. 

După luni de adunare a datelor, folosind sistemul implementat, am analizat datele, după care am făcut câteva observații interesante. Au fost mai multe modele după care se schimbau preturile, însă nu a fost niciunul concludent și valabil pentru toate produsele. Au fost mai multe cazuri unde am observat manipulări ale prețurilor, mai ales in perioada de Black Friday, dar pe de altă parte am observat și prețuri cu promoții reale și semnificative, așa că este destul de greu de generalizat în acest sens. După folosirea îndelungată a software-ului putem concluziona că acesta poate fi de folos unui potențial cumpărător, prin furnizarea unor informații utile și într-o formă ușor de citit, astfel economisind și timp. 

\textbf{Cuvinte cheie}: webscraping, extensie de browser, aplicație Flutter