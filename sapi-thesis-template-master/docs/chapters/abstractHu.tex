Napjainkban az emberek nagy többsége a vásárlásaikat az online térben bonyolítják le. Rengeteg webáruház létezik, szinte állandóan vannak kedvezmenyék vagy ajánlatok. Viszont sok esetben a feltüntetett árak ingadoznak, vagy a kedvezmény csak látszólagos. Ezekre akkor lehet felfigyelni, ha napi szinten követjük az árak alakulását, viszont ez időigényes és repetitív folyamat. Ugyanez elmondható akkor is, ha egy terméket megszeretnénk vásárolni kedvező áron. Egy másik népszerű online áruházakkal kapcsolatos jelenség a Black Friday napján történő drasztikus árcsökkentés, vagy legalábbis annak a látszata, hiszen sokszor hallottuk vagy gondoltuk, hogy ezek igazából hamisak vagy a csökkenés mértéke jóval kisebb, mint az valójában fel van tüntetve. Ezen gondolatok szolgálták a rendszer megvalósításához szükséges inspirációt.

A dolgozatban bemutatunk egy árakat követő rendszert, mely az előbb említett napi ellenőrzés folyamatát automatizálja. A rendszer meghatározott idő pillanatokban felébred, az online web áruházakon elérhető publikus adatokat bányássza, és az eredményeket elmenti. A rendszer biztosit egy böngésző kiegészítőt a termékek hozzáadására, követésére, illetve egy egyéb műveletekre. A mobil alkalmazás ugyanazt a funkcionalitást biztosítja, mint a böngésző kiegészítő, ez Flutter segítségével lett megvalósítva. A szoftver szívét úgymond egy Python script biztosítja, amely az adatok bányászását végzi a követni kívánt weboldalakról, valamint a periodikus frissítésért is felelős. Természetesen ezek az modszerek minden morális, illetve jogi feltételnek eleget tesznek a bányászás során. A bányászott adatok a Firebase által szolgáltatott Realtime Database segítségével vannak eltárolva, ami rendkívül jó és megbízható platformnak bizonyult a jelen felhasználásra.

A hónapokon keresztül gyűjtött adatokat kielemezve le tudtunk vonni pár következtetést éspedig azt, hogy valóban valamilyen szinten manipulálva vannak az árak, hiszen több érdekes mintára is felfigyeltünk, viszont ezek nem minden esetben a vásárló becsapását jelentik, volt példa reális és igencsak jelentős árcsökkenésre is, ezért eléggé nehéz általánosítani. A rendszer hosszabb idejű használata során az a következtetés vonódott le, hogy igencsak hasznos lehet egy ilyen jellegű szoftver a vásárlok számára, hiszen valamilyen mértékben könnyebbé teheti a döntési folyamatot, az információk összesítése és ábrázolása által.

\textbf{Kulcsszavak}: webscraping, böngésző kiegészítő, Flutter alkalmazás