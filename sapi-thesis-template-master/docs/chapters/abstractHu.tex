Napjainkban az emberek nagy többsége a vásárlásaikat az online térben bonyolítják le. Rengeteg webáruház létezik, szinte állandóan vannak kedvezmenyék vagy ajánlatok. Viszont sok esetben a feltüntetett árak ingadoznak, vagy a kedvezmény csak látszólagos. Ezekre akkor lehet felfigyelni, ha napi szinten követjük az árak alakulását, viszont ez időigényes és repetitív folyamat. Ugyanez elmondható akkor is, ha egy terméket megszeretnénk vásárolni kedvező áron. 
A dolgozatban bemutatunk egy árukövető rendszert, mely ezt a folyamatot automatizálja. 
A rendszer meghatározott idő pillanatokban felébred, az Interneten elérhető publikus adatokat bányássza, és az eredményeket elmenti. A felhasználónak egy böngésző kiegészítőt biztosítunk, mely segítségével hozzá adhat termékeket a követéshez a támogatott weboldalakról, illetve grafikonokon az ár változását is követheti. 
Ugyanakkor, mivel a vásárlások egyre nagyobb része zajlik telefonos alkalmazásokon keresztül, ezért ilyen platformra is elérhetővé tesszük a szolgáltatást egy Flutter segítségével készült alkalmazáson keresztül.

\textbf{Kulcsszavak}: webscraping, böngésző kiegészítő, Flutter alkalmazás